\section{Observability}
\subsection{State space matrices}
The system described in equations 13a-13f \cite{assignment} can be described in state space form 
with state \textbf{x} = $[\xi_w\: \psi_w\: \psi\; r\: b]^T$, input $u = \delta$, 
disturbances \textbf{w} = $[w_w\: w_b]^T$ and output $y$:

\begin{align*}
    \bf{\dot{x}} &= \textbf{Ax} + \textbf{B}u + \textbf{Ew} \\
    \textbf{y} &= \textbf{Cx}
\end{align*}

with the matrices:

\begin{equation}
    \label{eq:State Space With Disturbances}
    \begin{aligned}
        A &= 
    	\begin{bmatrix}
        0           &      1            & 0 & 0 			 & 0 \\
        -\omega_0^2 & -2\lambda\omega_0 & 0 & 0 			 & 0 \\
        0           & 0                 & 0 & 1              & 0 \\
    	0           & 0                 & 0 & -\frac{1}{T}   & -\frac{K}{T} \\
    	0           & 0                 & 0 & 0 			 & 0 \\
      \end{bmatrix}
      \qquad
      B =
      \begin{bmatrix}
        0 \\
        0 \\
    	0 \\
        \frac{K}{T} \\
        0 \\
      \end{bmatrix}
      \\
      E &=
      \begin{bmatrix}
        0   & 0 \\
        K_w & 0 \\
    	0   & 0 \\
        0   & 0 \\
        0   & 1 \\
      \end{bmatrix}
      \qquad
      C =
      \begin{bmatrix}
        0 & 1 & 1 & 0 & 0 \\
      \end{bmatrix}
    \end{aligned}
\end{equation}

\subsection{Without disturbances}
For Part B-E, the observability matrix is used to study observability. It is defined as:

\begin{equation}
  \bm{\mathcal{O}} =
  \begin{bmatrix}
    \bm{C} \\
    \bm{CA} \\
    \bm{C*A^2} \\
    \vdots \\
    \bm{C*A^{n-2}} \\
    \bm{C*A^{n-1}} \\
  \end{bmatrix} \\
\end{equation}

where n is the number of state variables. The observability matrix is calculated using MATLAB's \texttt{obsv(A, C)} command.  If the matrix is full column rank, then the system is observable.

When examining observability without disturbances, the states affected by disturbances are disregarded . The remaining states are $\psi$ and r. With these states, the state space matrices become:

\begin{equation}
    A = 
	\begin{bmatrix}
    0 & 1            \\
	0 & -\frac{1}{T} \\
	\end{bmatrix}
  \qquad
  B =
  \begin{bmatrix}
	0 \\
    \frac{K}{T} \\
  \end{bmatrix}
  \qquad
  C =
  \begin{bmatrix}
    1 & 0\\
  \end{bmatrix}
\end{equation}

and the observability matrix becomes:

\begin{equation}
  \bm{\mathcal{O}} =
  \begin{bmatrix}
    1 & 0 \\
    0 & 1 \\
  \end{bmatrix}
\end{equation}

The rank of this matrix is 2, which is full rank, therefore the system without disturbances is observable.

\subsection{With current}
When examining observability with only disturbances from the current, the states affected by wave disturbances are disregarded. The remaining states are $\psi$, r and b. With these states, the state space matrices become:

\begin{equation}
    A = 
	\begin{bmatrix}
    0 & 1            & 0\\
	0 & -\frac{1}{T} & -\frac{K}{T}\\
	0 & 0            & 0
	\end{bmatrix}
  \qquad
  B =
  \begin{bmatrix}
	0 \\
    \frac{K}{T} \\
    0 \\
  \end{bmatrix}
  \qquad
  C =
  \begin{bmatrix}
    1 & 0 & 0\\
  \end{bmatrix}
\end{equation}

and the observability matrix becomes:

\begin{equation}
  \bm{\mathcal{O}} =
  \begin{bmatrix}
    1 & 0       & 0\\
    0 & 1       & 0\\
    0 & -0.0117 & -0.002 \\
  \end{bmatrix}
\end{equation}

The rank of this matrix is 3, which is full rank, therefore the system with only current disturbances is observable.

\subsection{With waves}
When examining observability with only disturbances from the waves, the states affected by wave disturbances are disregarded. The remaining states are $\xi_w$, $\psi_w$, $\psi$ and r. With these states, the state space matrices become:

\begin{equation}
    A = 
	\begin{bmatrix}
    0           & 1                 & 0 & 0\\
	-\omega_0^2 & -2\lambda\omega_0 & 0 & 0\\
	0           & 0                 & 0 & 1\\
	0           & 0                 & 0 & -\frac{1}{T}\\
	\end{bmatrix}
  \qquad
  B =
  \begin{bmatrix}
	0 \\
    0 \\
    0 \\
    \frac{K}{T}
  \end{bmatrix}
  \qquad
  C =
  \begin{bmatrix}
    0 & 1 & 1 & 0\\
  \end{bmatrix}
\end{equation}

and the observability matrix becomes:



\begin{equation}
  \bm{\mathcal{O}} =
  \begin{bmatrix}
    0      &  1       & 1   & 0 \\
   -0.6120 &  -0.1294 & 0   & 1 \\
    0.0792 &  -0.5953 & 0   & -0.0117 \\
    0.3643 &   0.1562 & 0   & 0.0001 \\
  \end{bmatrix}
\end{equation}

The rank of this matrix is 4, which is full rank, therefore the system with only wave disturbances is observable.
\subsection{With current and waves}
The system with both disturbances from the current and from the waves has the state space form described in Part 4.1 \cref{eq:State Space With Disturbances}.

The observability matrix for this system is:

\begin{equation}
  \bm{\mathcal{O}} =
  \begin{bmatrix}
         0  &       1  &       1 &        0 &  0 \\
   -0.6120  & -0.1294  &       0 &        1 &  0 \\
    0.0792  & -0.5953  &       0 &  -0.0117 &  -0.0020 \\
    0.3643  &  0.1562  &       0 &   0.0001 &  0 \\
   -0.0956  &  0.3441  &       0 &        0 &  0 \\
  \end{bmatrix}
\end{equation}

The rank of this matrix is 5, which is full rank, therefore the system with both current and wave disturbances is observable.
